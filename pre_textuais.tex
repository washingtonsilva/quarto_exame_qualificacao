
\thispagestyle{empty}

\begin{center}
    \large
    Instituto Federal de Educação, Ciência e Tecnologia de Minas Gerais \\ 
    \textit{Campus Formiga} \\
    \sc Mestrado Profissional em Administração
    
    \vspace{3cm}
    JULIANO MENDONÇA TERRA
    
    \vspace{3cm}
    \textbf{GESTÃO DE RISCO NO SETOR PÚBLICO:} \\
    Subtítulo
    
    \vspace{2cm}
    \begin{flushright}
    \begin{minipage}{0.6\textwidth}
    \small
    Projeto apresentado ao Programa de Pós-
    Graduação em Administração do Instituto
    Federal de Educação, Ciência e
    Tecnologia de Minas Gerais (IFMG) -
    \textit{Campus} Formiga, como requisito para
    obtenção de aprovação na Qualificação
    do Mestrado.
    \end{minipage}
    \end{flushright}
    
    \vspace{0.5cm}
    \begin{flushright}
    \small
    Orientador: Prof. Dr. Washington Santos da Silva.\\
    Coorientador: Prof. Dr. Lélis Pedro de Andrade.
    
    \vspace{0.5cm}
    Linha de Pesquisa: Finanças Corporativas e Investimentos.
    \end{flushright}
    
    \vfill
    Formiga, Minas Gerais \\
    2025
\end{center}


\newpage

\thispagestyle{empty}

\newenvironment{meuresumo}{
  \clearpage
  \small
  \vspace{-1cm}
  \begin{center}
    \bfseries RESUMO
    \vspace{0.5em}
  \end{center}
  \begin{quote}
}{
  \end{quote}
  \vspace{-1.1em}
  \begin{center}
  \begin{minipage}{0.87\textwidth} 
  \textbf{Palavras-chave:} palavra 1, palavra 2, palavra 3.
  \end{minipage}
  \end{center}
  \clearpage
}

\begin{meuresumo}
O objetivo deste relatório técnico conclusivo foi identificar o estado 
da arte das técnicas e variáveis utilizadas no desenvolvimento de 
modelos de previsão do risco de crédito baseados em algoritmos de 
aprendizagem de máquina. Com base em uma revisão sistemática da 
literatura recente, abrangendo estudos publicados entre 2019 e 2023, 
foram identificados os modelos mais utilizados, as métricas de 
avaliação de desempenho e as principais limitações dos dados 
informadas por pesquisadores da área. Constatou-se o uso tanto 
de modelos híbridos quanto de não híbridos, destacando a eficácia 
dos modelos que utilizam o método \textit{boosting} para melhoria 
da acurácia preditiva. Foram também mapeados os grupos e as variáveis 
mais frequentemente utilizadas, além das técnicas de pré-processamento 
aplicadas para lidar com desbalanceamento de dados e determinação 
de hiperparâmetros. O relatório conclui com recomendações práticas 
para o desenvolvimento de modelos preditivos do risco de crédito, 
enfatizando a importância de iniciar com modelos tradicionais 
para estabelecer benchmarks e, posteriormente, incorporar modelos 
híbridos mais sofisticados. Além disso, recomenda-se a implementação 
de processos de reavaliação periódica dos dados e dos modelos para 
garantir a eficácia das previsões ao longo do tempo.
\end{meuresumo}




% formatação do título de cada seção
\makeatletter
\def\@makechapterhead#1{%
  \vspace*{50\p@}%
  {\parindent \z@ \raggedright \normalfont
    \ifnum \c@secnumdepth >\m@ne
      \huge\bfseries \thechapter\space
    \fi
    \huge \bfseries #1\par\nobreak
    \vskip 40\p@
  }}
\makeatother


